Homework 1


\begin{frame}{518 only}
    1 page - tie your research interests to the Astro2020 Decadal or the PlanetaryDecadal2022 surveys and list possible targets for an observing proposal that relates to priority science. 
    This does not need to be an original experiment, re-observing targets or reproducing past work that the Decadal rests on is \textit{preferred}.
    You are encouraged to reach out to scientists in Steward and LPL: \url{https://www.as.arizona.edu/people/faculty}.
    You don't need to worry about the observatory capabilities, yet.
\end{frame}
\begin{frame}{Frame Title}
Derive the thin lens formula.

\end{frame}
\begin{frame}{Observing Proposal Prep}
    
The Steward Observatory 2.3 m ``Bok'' telescope is of Ritchey-Cretién design.% with a scale at the focal plane of 10”/mm. 
 Details: \url{https://lavinia.as.arizona.edu/~tscopewiki/doku.php?id=public:kitt_peak:bok_90:bok_90_telescope}
 
 \begin{enumerate}
\item Calculate the f-number of secondary mirror when the telescope is f/9
\item Calculate the plate scale at prime focus
\item What is the distance from the primary to prime focus?
\item draw these configurations of the telescope with distances labelled
\end{enumerate}
\end{frame}

\begin{frame}{Point source part I)}
Williams et al 2004 describes the 90 prime instrument and Lesser and Vu 2004 describe the sensors:
\begin{enumerate}
    \item Find the typical read noise and gain
    \item Find the typical dark noise
    \item  estimate by eye the average QE in V-band
    \item estimate by eye the reflectivity of the aluminum primary mirror
    \item count the number of optics and calculate the system throughput assume 2\% anti-reflection coating losses and an aluminum coated primary mirror
    \item Calculate the scattered light background per pixel and per 1 arcsec sq. PSF for a solar zodiacal background of 22 magnitudes/as$^2$ and no moonshine. 
    \item calculate the time to SNR for a seeing limited 15th magnitude point source, including the terms above
\end{enumerate}
    
\end{frame}

\begin{frame}{Point source part II)}

Identify a point source of interest and it's key parameters, using Simbad or JPL Ephemeris and any other resources necessary:

 \begin{enumerate}
 \item Look up the target on Simbad: \url{simbad.cds.unistra.fr/simbad}
 \item Calculate the flux from the target in photons/cm$^2/\AA $/sec in the band of interest
 \item estimate the exposure time necessary to reach SNR=5 and SNR=100
 \item if this exposure time is impractical ($\gtrsim$2hrs), revisit your assumptions or target and repeat
 \item Why would you want to reach SNR 100 rather than 5?
 \item list any special considerations
\end{enumerate}
\end{frame}

\begin{frame}{Extended source}


 \begin{enumerate}
 \item
\end{enumerate}
\end{frame}



\begin{frame}{Source availability}


 \begin{enumerate}
 \item
\end{enumerate}
\end{frame}


