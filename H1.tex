\section{HW1}
\begin{frame}{Book problems}
\begin{enumerate}
  %  \item Show that "brightness is conserved" by finding the flux from a source radius R at distance d, emitting P photons/second. What is the rate of photons per pixel detected by telescope diameter D with pixel size $\theta$ if the source is A) unresolved, or b) resolved  %inspired by R+L 1.3
    \item Rieke, Measuring the Universe, problems 1.1 and 1.5
\end{enumerate}
\end{frame}

\begin{frame}{Think big}
    1 page - tie your research interests to the Astro2020 Decadal or the PlanetaryDecadal2022 surveys and list possible targets for an observing proposal that relates to priority science. 
    This does not need to be an original experiment, re-observing targets or reproducing past work that the Decadal rests on is \textit{preferred}.
    You are encouraged to reach out to scientists in Steward and LPL: \url{https://www.as.arizona.edu/people/faculty}, \url{https://www.lpl.arizona.edu/faculty}.
    You don't need to worry about the observatory capabilities, yet.
\end{frame}

\begin{frame}
  \frametitle{Flux from the Sun}

1.  Neglecting small scale variations such as sunspots, granules, and faculae, assume the Sun radiates isotropically. The Sun has radius $R_\sun$.
 
(a)      How does the flux $f$ depend on distance from the sun ($d$)? Recall the total power of the sun is its luminosity $L$.

(b)       For an observer at a distance $d$ >> $R_\sun$, what solid angle, $\Omega$, does the Sun 
appear or subtend?  
 
(c)     Calculate the observed surface brightness ($B$) as the ratio of the observed flux to the observed solid angle ($f/\Omega$)
 
(d)      How does $B$ depend on $r$? Why is this a useful quantity? 

(e) Define the word ``conserve'' in the physical sense. Is $B$ conserved?

(f) Look up the value for $L$ and assume $d$ is 1 astronomical unit. What is the flux at the Earth? 
(g) Look up the value for $L$ and assume $d$ is 10 parsec. What is the flux at the Earth? 
\end{frame}
