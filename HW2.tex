
\section{HW 2}


\begin{frame}{Echelle Spectroscopy}

 \begin{enumerate}
 \item Consider the design of a diffraction grating spectrometer for a 10 m telescope. The 2-pixel resolution
element is 0.5” and the required resolving power is R=20,000. Assume that the configuration is Littrow.
Two gratings are available, a first-order grating blazed at 17.5° and an echelle grating blazed at 63.5°.
Determine the collimator size required in both cases. Which is more practical?
\item Assuming the telescope
provides a F/15 beam, what is the focal length of these two collimators?
\end{enumerate}
\end{frame}


%\begin{frame}{Basic Questions}
% \begin{enumerate}
% \item Derive the thin lens formula.
%\end{enumerate}

%\end{frame}


\begin{frame}{Spatial Frequencies}
0) draw cross-sections of perfect reflectivity  (1) and perfect optical path error versus distance across the optic (0 nm RMS) for an ideal D=8.4 m telescope primary mirror.

a)	Assume a 8.4 cm pitch lap runs is miscalibrated and adds a sinusiodal 100 nm RMS wavefront error at a spatial frequency equal to one over its size, and just along one axis. Draw new a new cross section of the OPD along the axis of the error and label the peak-to-valley error.
b)	Draw a two dimensional representation of the mirror with this error
c)	Calculate the Fourier series or estimate it by eye and draw the resultant 1D PSF cross section and sketch a 2D representation of this new PSF at 1 micron. Label both drawings in arcseconds and units of l/D. l is a wavelength of 1 micron.
d)	In signal processing the Nyquist frequency, one half the sampling rate of a system, is the minimum frequency that can be measured by the system. By this analogy, how many wavefront sensor elements and deformable mirror actuators would be required to correct this primary mirror error?
\end{frame}
\begin{frame}{}
\end{frame}