\section{Observing Project}


\begin{frame}{Observing Proposal Prep}
    
The Steward Observatory 2.3 m ``Bok'' telescope is of Ritchey-Cretién design.% with a scale at the focal plane of 10”/mm. 
 Details: \url{https://lavinia.as.arizona.edu/~tscopewiki/doku.php?id=public:kitt_peak:bok_90:bok_90_telescope}
 
 \begin{enumerate}
 \item derive the thin lens equation
\item Calculate the f-number of secondary mirror when the telescope is f/9
\item Calculate the plate scale at prime focus
\item What is the distance from the primary to prime focus?
\item draw these configurations of the telescope with distances labelled
\end{enumerate}
\end{frame}

\begin{frame}{Point source part I)}
Williams et al 2004 describes the 90 prime instrument and Lesser and Vu 2004 describe the sensors:
\begin{enumerate}
    \item Find the typical read noise and gain
    \item Find the typical dark noise
    \item  estimate by eye the average QE in V-band
    \item estimate by eye the reflectivity of the aluminum primary mirror
    \item count the number of optics and calculate the system throughput, assume 2\% anti-reflection coating losses and an aluminum coated primary mirror
    \item Calculate the scattered light background per pixel and per 1 arcsec sq. PSF for a solar zodiacal background of 22 magnitudes/as$^2$ and no moonshine. 
    \item calculate the time to SNR for a seeing limited 15th magnitude point source, including the terms above
\end{enumerate}
    
\end{frame}

\begin{frame}{Point source part II)}

Identify a point source of interest and its key parameters, using Simbad (\url{simbad.cds.unistra.fr/simbad}) or JPL Ephemeris and any other resources necessary:

 \begin{enumerate}
\item describe the targets key parameters, including at least V-mag, RA, and DEC
\item Calculate the flux from the target in photons/cm$^2/\AA $/sec in the band of interest
 \item estimate the exposure time necessary to reach SNR=5 and SNR=100
 \item if this exposure time is impractical ($\gtrsim$2hrs), revisit your assumptions or target and repeat
 \item Why would you want to reach SNR 100 rather than 5?
 \item list any special considerations
\end{enumerate}
\end{frame}

\begin{frame}{Extended source}
Same as point source but for an extended source (solar system through extra-galactic is acceptable).

 \begin{enumerate}
 \item repeat point source but add surface brightness.
\end{enumerate}
\end{frame}





\begin{frame}{Source availability}

 \begin{enumerate}
 \item install astroplan: \url{https://github.com/astropy/astroplan}
 \item run the basic tutorials: \url{https://astroplan.readthedocs.io/en/latest/tutorials/summer_triangle.html}
 \item adjust the location to KPNO and your selected targets
 \item create an observing plan that allows you to get to your targets, based on the time to SNR from the previous problem
 \item make a plot that shows all your targets of interest in airmass vs local time
 \item form an ad-hoc telescope allocation committee with other students on your night(s) and generate an observing plan that allows everyone to observe, revising your targets and exposures times as necessary 
 \item  assume you might have to adapt the plan without internet:
 \begin{itemize}
    \item write the plan and key parameters in your notebook
     \item print out finder charts
 \end{itemize}
\end{enumerate}
\end{frame}



\begin{frame}{Code review}

you will be randomly assigned a classmates SNR code calculation code to review. 
Review the Google Code Review docs: \url{https://google.github.io/eng-practices/review/reviewer/looking-for.html}

Concentrate on:
 \begin{enumerate}
 \item does the code run?
 \item is it clear what each line does and why?
 \item does it follow the python style guide: \tiny{\url{https://peps.python.org/pep-0008/\#a-foolish-consistency-is-the-hobgoblin-of-little-minds}}
\end{enumerate}

\end{frame}


\begin{frame}{Observing Report - 518 only}
 \begin{enumerate}
 \item Use AASTex \url{https://journals.aas.org/aastexguide/} 
 \item motivate the measurement with an explicit citation to a decadal survey.
 \item substantiate or cite any assumptions
 \item present equations even if they appear trivial
\item Report measured read noise (electron/sec), bias (ADU), dark rate (electrons/sec), and gain (electrons/ADU).
\item Report measured count rate from your point source and extended source
\item Report SNR of both a point source and an extended source
\item report photometric brightness of your sources in Janskys and magnitudes (per sq arcsec for the extended source)
\item suggest future improvements and astrophysical significance in your conclusion
\end{enumerate}
\end{frame}